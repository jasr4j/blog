\documentclass{beamer}
\usetheme{Boadilla}
\newcommand{\ba}{\overline}
\definecolor{resulthead}{RGB}{205,205,235}
\definecolor{termshead}{RGB}{242,218,195}
\begin{document}

\title[Math and Proofs]{Math and Proofs Class 3}
\date{September 26th, 2017}

\begin{frame}[plain]
\titlepage
\end{frame}

\begin{frame}{Fun aside: Euclid's Proof that there are infinitely many prime numbers}
\begin{itemize}
\item A prime number is a number with only two factors: 1 and itself
\item Every number is made up of prime numbers multiplied together
\item So maybe there are finitely many of them? We can multiply them together in so many different ways. Maybe we can create all the numbers in that way?
\item Nope! Multiply them all together and add 1. What does this do?
\end{itemize}
\end{frame}

\begin{frame}{Recap of Last Class}
\begin{itemize}
\item We started to learn about set theory
\item We looked at some set theory operations and did some simple proofs
\end{itemize}
\end{frame}

\begin{frame}{Set Theory Reminders}
\begin{itemize}
\item Empty Set: $\emptyset$
\item Union: $A\cup B$
\item Intersection: $A\cap B$
\item Set minus: $A\setminus B$ (the elements in $A$ but not $B$)
\item Subset: $A\subseteq B$
\item Element: $x\in A$
\end{itemize}
\end{frame}

\begin{frame}{Set Theory Results}
\begin{enumerate}
\item For any set $A$, $\emptyset \subseteq A$
\item For any set $A$, $A\subseteq A$
\item If $A$, $B$, and $C$ are sets where $A\subset B$ and $B\subset C$, then $A\subset C$
\item Let $A$ and $B$ be sets. Then $A=B$ if and only if $A\subset B$ and $B\subset A$.
\end{enumerate}
\end{frame}

\begin{frame}{Exercise 1}
Let $A = \{x,y,\{x,y\}\}$. True or false:
\begin{enumerate}
\item $\{x,y\}\subseteq A$
\item $\{x,y\}\in A$
\item $\{y\}\subseteq A$
\item $\{y\}\in A$
\end{enumerate}
\end{frame}

\begin{frame}{Exercise 2}
Let $A = \{1,2,3\}, B = \{2,3,4\}, C = \{4,5,6\}$. Find each of the following sets:
\begin{enumerate}
\item $A\cup B$
\item $A\cap B$
\item $B\cap C$
\item $A\setminus B$
\item $A\cap (B\cup C)$
\item $(A\cap B)\cup (A\cap C)$
\end{enumerate}
\end{frame}

\begin{frame}{More Set Theory Results}
Let $A,B,C$ be sets
\begin{enumerate}
\item $A\cap (B\cup C) = (A\cap B)\cup (A\cap C)$
\item $A\cup (B\cap C) = (A\cup B)\cap (A\cap C)$ (exercise)
\end{enumerate}
\end{frame}

\begin{frame}{Cartesian Product}
The Cartesian Product of two sets $A$ and $B$ is the set of ordered pairs with first entry in $A$ and second entry in $B$.
\\\vspace{10pt}
Example: $A = \{1,2\}, B = \{dog, cat, child\}$. Then $A\times B = \{(1,dog), (1,cat), (1,child), (2,dog), (2,cat), (2,child)\}$.
\\\vspace{10pt}
Questions:
\begin{enumerate}
\item If $A = \{a,b\}, B = \{34\}$, what are $A\times B$ and $B\times A$?
\item If $A$ has 4 elements and $B$ has 3 elements, how many elements does $A\times B$ have?
\item If $\mathbb{R}$ is the set of ``real'' numbers, you could say that $\mathbb{R}$ represents the number line. In this way of thinking, what is $\mathbb{R}\times\mathbb{R}$? 
\end{enumerate}
\end{frame}

\begin{frame}{Power Sets}
If $A$ is a set, then the power set $P(A)$ is the set of all subsets of $A$.
\\\vspace{10pt}
Example: $A = \{a,b,c\}$. $P(A) = \{\emptyset, \{a\}, \{b\}, \{c\}, \{a,b\}, \{a,c\}, \{b,c\}, \{a,b,c\}\}$.
\\\vspace{10pt}
Exercises:
\begin{enumerate}
\item If $A = \{1,2\}$, what is $P(A)$?
\item Find the power set of the empty set
\item Prove:
\begin{itemize}
\item If $A\subset B$, then $P(A)\subset P(B)$
\item If $A\cap B = \emptyset$, then $P(A)\cap P(B) = \emptyset$
\end{itemize}
\end{enumerate}
\end{frame}

\begin{frame}{Next Time}
\begin{itemize}
\item We'll look at functions, bijections, and equivalence relations
\item In two weeks, we'll start looking at cardinality and infinity
\end{itemize}
\end{frame}

\end{document}