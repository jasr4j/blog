\documentclass{beamer}
\usepackage{graphicx}
\usetheme{Boadilla}
\newcommand{\ba}{\overline}
\definecolor{resulthead}{RGB}{205,205,235}
\definecolor{termshead}{RGB}{242,218,195}
\begin{document}

\title[Math and Proofs]{Math and Proofs Class 8}
\date{November 14th, 2017}

\begin{frame}[plain]
\titlepage
\end{frame}

\begin{frame}{Summary of the class so far}
\begin{itemize}
\item First two weeks: looked at a couple different axiom systems
\item Since then: set theory
\item Last time: talked about the Axiom of Choice
\item Goodstein's Theorem
\begin{itemize}
\item Can be \textit{stated} using the Peano axioms (normal arithmetic)
\item Can only be \textit{proven} using ordinals (which are \textit{not} part of normal arithmetic)
\end{itemize}
\end{itemize}
\end{frame}

\begin{frame}{Hilbert's Program}
\begin{itemize}
\item Can we build a complete axiom system?
\item ``Complete'' means that everything we can state using our axioms we can prove true or false using our axioms
\begin{itemize}
\item We don't have to prove \textit{everything} true or false, just everything we can state using the axioms
\item For instance, ``I like ice cream'' is not a likely candidate for proof in any axiom system we come up with, but that's OK because we can't state it in our usual axiom systems
\end{itemize}
\item So in a sense, it's really easy to make a complete axiom system (no axioms or clock math)
\item But we want axioms that are powerful, like our normal math
\end{itemize}
\end{frame}

\begin{frame}{Hilbert's Program (cont.)}
\begin{itemize}
\item In 1900, David Hilbert published a list of 23 open problems throughout mathematics that he considered to be the most important problems of the day
\item Even now, many have not been solved
\item Hilbert's Second Problem: Prove that the axioms of arithmetic are consistent (no contradictions)
\end{itemize}
\end{frame}

\begin{frame}{Hilbert's Program (cont.)}
\begin{itemize}
\item Further, Hilbert envisioned a program to secure the foundations of mathematics, namely (copied from Wikipedia):
\begin{itemize}
\item A formulation of all mathematics; in other words all mathematical statements should be written in a precise formal language, and manipulated according to well defined rules.
\item Completeness: a proof that all true mathematical statements can be proved in the formalism.
\item Consistency: a proof that no contradiction can be obtained in the formalism of mathematics. This consistency proof should preferably use only ``finitistic'' reasoning about finite mathematical objects.
\item Conservation: a proof that any result about ``real objects'' obtained using reasoning about ``ideal objects'' (such as uncountable sets) can be proved without using ideal objects.
\item Decidability: there should be an algorithm for deciding the truth or falsity of any mathematical statement.
\end{itemize}
\end{itemize}
\end{frame}

\begin{frame}{Godel's Incompleteness Theorem}
\begin{itemize}
\item Kurt Godel (1931): Hilbert's program is impossible
\item There is no consistent system of axioms that is capable of proving true or false all statements about the natural numbers
\item What this means: there are tons and tons more ``theorems'' like Goodstein's Theorem, that need extra axioms to prove
\item No axiom system can prove its own consistency
\end{itemize}
\end{frame}

\begin{frame}{Godel's Incompleteness Theorem (cont.)}
\begin{itemize}
\item Example: The axiom of choice cannot be proved or disproved
\item Example: The continuum hypothesis cannot be proved or disproved (whether we assume the axiom of choice or not)
\item Proof idea: encode statement as symbols, and use that to encode statements as numbers
\item Define a statement that roughly means ``I am not provable'' (not easy to do, but Godel did it)
\item This is a paradox, which means that not every statement can be proved or disproved
\end{itemize}
\end{frame}

\begin{frame}{Implications from Godel's Incompleteness Theorem}
\begin{itemize}
\item We can never make an axiom system that can prove everything we want it to prove
\item We can also never prove that an axiom system is consistent
\item So we need to be really careful in choosing our axioms
\item However, we can use a more sophisticated viewpoint to make this much less of a problem than it seems. If we assume certain things about ordinals, then we can use this to prove the consistency of our axioms
\item So as long as these ordinal assumptions are correct, we don't have to worry
\end{itemize}
\end{frame}

\begin{frame}{Conclusion}
\begin{itemize}
\item What this \textit{doesn't} mean is that math is useless or wrong
\item Math is still the most successful and accurate way we have of understanding the universe
\item It is highly likely that our axioms are consistent; we just can never prove it
\item We may just need to think of it more as a science, instead of some greater form of understanding, and with this viewpoint, it's by far the most accurate science of them all
\end{itemize}
\end{frame}


\end{document}