\documentclass{beamer}
\usepackage{graphicx}
\usetheme{Boadilla}
\newcommand{\ba}{\overline}
\definecolor{resulthead}{RGB}{205,205,235}
\definecolor{termshead}{RGB}{242,218,195}
\begin{document}

\title[Math and Proofs]{Math and Proofs Class 7}
\date{November 7th, 2017}

\begin{frame}[plain]
\titlepage
\end{frame}

\begin{frame}{More about the Axiom of Choice}
\begin{itemize}
\item \textit{The Axiom of Choice is obviously true, the Well-Ordering Principle is obviously false, and who can tell about Zorn's Lemma?}
\item Axiom of Choice: If we have infinitely many buckets, we can form a set with one item from each bucket
\item Zorn's Lemma: If $S$ is any nonempty partially ordered set in which every chain has an upper bound, then $S$ has a maximal element. (don't worry about this one too much)
\item Well-Ordering Principle: Every set can be ``well-ordered''
\end{itemize}
\end{frame}


\begin{frame}{Ordinal Numbers}
\begin{itemize}
\item Kind of like counting numbers, but can be infinite
\item Two ways of getting ordinals
\begin{enumerate}
\item Successor
\item Limit
\end{enumerate}
\end{itemize}
\end{frame}

\begin{frame}{Weak Goodstein's Theorem}
\begin{itemize}
\item Procedure for the Goodstein sequence with starting point $n$:
\begin{enumerate}
\item $g_1=n$
\item Write this number in base-$2$ notation
\item Change all the 2's to 3's
\item Subtract 1. This is $g_2$
\item Continue to find $g_3, g_4,\ldots$.
\end{enumerate}
\item Example: starting point 5
\begin{enumerate}
\item $g_1 = 5$
\item This equals $2^2 + 1$
\item Change it to $3^2 + 1 (= 10)$
\item Now subtract 1 from that: $g_2 = 3^2$
\end{enumerate}
\item We can prove that this sequence always reaches 0 using ordinal numbers
\end{itemize}
\end{frame}

\begin{frame}{Goodstein's Theorem}
\begin{itemize}
\item Now we write numbers in their ``hereditary'' representation
\item Example: starting point 33
\begin{enumerate}
\item $g_1 = 33 = 2^{2^2+1} + 1$
\item Change 2's to 3's: $3^{3^3+1} + 1$
\item Then subtract 1: $g_2 = 3^{3^3+1} = 3^10 = 59049$
\item $g_3$ starts with 5, and has 155 digits!
\end{enumerate}
\item What's different here from the ``weak'' case (besides being harder)?
\end{itemize}
\end{frame}

\begin{frame}{Next Time}
\begin{itemize}
\item The End: Hilbert's Program: Can we build a complete axiom system? 
\end{itemize}
\end{frame}

\end{document}