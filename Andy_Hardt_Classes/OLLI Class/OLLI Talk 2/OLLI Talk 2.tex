\documentclass{beamer}
\usetheme{Boadilla}
\newcommand{\ba}{\overline}
\definecolor{resulthead}{RGB}{205,205,235}
\definecolor{termshead}{RGB}{242,218,195}
\begin{document}

\title[Math and Proofs]{Math and Proofs Class 2}
\date{September 26th, 2017}

\begin{frame}[plain]
\titlepage
\end{frame}

\begin{frame}{Recap of Last Class}
\begin{itemize}
\item We looked at 2 examples of axiom systems
\begin{itemize}
\item Euclidean Geometry
\item Peano Axioms of Arithmetic
\end{itemize}
\item Left off before proving that addition is commutative
\end{itemize}
\end{frame}

\begin{frame}{Peano Axioms}
\begin{enumerate}
\item Zero is a number
\item If $a$ is a number, the successor of $a$ is a number
\item Zero is not the successor of a number
\item Two numbers of which the successors are equal are themselves equal
\item If a set $S$ of numbers contains zero and also the successor of every number in $S$, then every number is in $S$.
\end{enumerate}
\end{frame}

\begin{frame}{Set Theory}
\begin{itemize}
\item System we use today
\item Pioneered by Georg Cantor in the 1890s
\item Kronecker: \textit{I don't know what predominates in Cantor's Theory -- philosophy or theology, but I am sure that there is no mathematics there}
\item Feferman: \textit{Simply not relevant to everyday mathematics}
\item Hilbert: \textit{No one will drive us from the paradise which Cantor created for us}
\item Subsumes both Euclidean geometry and the Peano axioms (and much else)
\item (Much of this material is taken from \textit{Doing Mathematics} by Steven Galovich and from \textit{An Outline of Set Theory} by James M. Henle)
\end{itemize}
\end{frame}

\begin{frame}{Set Theory (cont.)}
\begin{itemize}
\item Undefined terms: set, element
\item Axioms: ZFC
\begin{itemize}
\item Equality: Two sets are equal if and only if they have the same elements
\item Empty Set: There is a set with no elements (called the \emph{empty set}: $\emptyset$)
\item Union: If $d$ is a set of sets, then the union of these sets is a set
\item Power Set: If $d$ is a set, then the collection of all subsets of $d$ is also a set (called the power set: $\mathcal{P}(d)$)
\item Many more, some of which are quite technical
\end{itemize}
\item Notation: $=, $\{\}$, \in, \subseteq, \emptyset, \cup, \cap, \setminus, \mathcal{P}(d)$
\end{itemize}
\end{frame}

\begin{frame}{Set Theory Results}
\begin{enumerate}
\item For any set $A$, $\emptyset \subseteq A$
\item For any set $A$, $A\subseteq A$
\item If $A$, $B$, and $C$ are sets where $A\subset B$ and $B\subset C$, then $A\subset C$
\item Let $A$ and $B$ be sets. Then $A=B$ if and only if $A\subset B$ and $B\subset A$.
\end{enumerate}
\end{frame}

\begin{frame}{Exercise 1}
Let $A = \{x,y,\{x,y\}\}$. True or false:
\begin{enumerate}
\item $\{x,y\}\subset A$
\item $\{x,y\}\in A$
\item $\{y\}\subset A$
\item $\{y\}\in A$
\end{enumerate}
\end{frame}

\begin{frame}{Exercise 2}
Let $A = \{1,2,3\}, B = \{2,3,4\}, C = \{4,5,6\}$. Find each of the following sets:
\begin{enumerate}
\item $A\cup B$
\item $A\cap B$
\item $B\cap C$
\item $A\setminus B$
\item $A\cap (B\cup C)$
\item $(A\cap B)\cup (A\cap C)$
\end{enumerate}
\end{frame}

\begin{frame}{More Set Theory Results}
Let $A,B,C$ be sets
\begin{enumerate}
\item $A\cap (B\cup C) = (A\cap B)\cup (A\cap C)$
\item $A\cup (B\cap C) = (A\cup B)\cap (A\cap C)$ (exercise)
\end{enumerate}
\end{frame}

\begin{frame}{Next Time}
\begin{itemize}
\item More set theory!
\end{itemize}
\end{frame}

\end{document}