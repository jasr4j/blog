\documentclass[12pt]{article}
\usepackage{fullpage, amsmath, amssymb, amsthm, amscd, setspace, bm, graphicx, indentfirst, multirow, tikz, enumerate, verbatim, appendix}
\usepackage{adjustbox,amsfonts,array,graphicx,booktabs,tabularx,multirow,multicol,stmaryrd, tabu}
\usepackage[utf8]{inputenc}
\usepackage{cite}
\usepackage{hyperref}
\usetikzlibrary{arrows.meta}
\title{Math 506, Spring 2026 -- Homework 2}
\date{}
\setlength{\parskip}{0.5cm}
\setlength{\parindent}{0cm}

\newtheorem{theorem}{Theorem}[section]
\newtheorem{definition}[theorem]{Definition}
\newtheorem{lemma}[theorem]{Lemma}
\newtheorem{proposition}[theorem]{Proposition}
\newtheorem{corollary}[theorem]{Corollary}
\newtheorem{remark}[theorem]{Remark}
\newtheorem{example}[theorem]{Example}

\newcommand{\Z}{\mathbb{Z}}
\newcommand{\R}{\mathbb{R}}
\newcommand{\Q}{\mathbb{Q}}
\newcommand{\C}{\mathbb{C}}
\newcommand{\N}{\mathbb{N}}
\renewcommand{\t}[1]{\text{#1}}
\newcommand{\ba}{\overline}
\newcommand{\Tr}{\text{Tr}}
\newcommand{\Hom}{\text{Hom}}
\newcommand{\End}{\text{End}}
\newcommand{\Aut}{\text{Aut}}
\newcommand{\Mat}{\text{Mat}}
\newcommand{\Sym}{\text{Sym}}
\newcommand{\Res}{\text{Res}}
\newcommand{\Ind}{\text{Ind}}
\newcommand{\ind}{\text{ind}}
\newcommand{\wt}{\text{wt}}

\makeatletter
\newcommand{\Wedge}{\@ifnextchar^\@Wedge{\@Wedge^{\,}}}
\def\@Wedge^#1{\mathop{\bigwedge\nolimits^{\!#1}}}
\makeatother

\begin{document} \maketitle
\vspace{-80pt}

\textbf{Due:} Wednesday, February 25th, at 9:00am via Gradescope.

\textbf{Instructions:} Students should complete and submit all problems. All assertions require proof unless otherwise stated. Typesetting your homework using LaTeX is recommended.

For this homework, unless otherwise stated all groups are finite and all representations are finite dimensional and complex.

\begin{enumerate}

\item[1.] Let $V$ and $W$ be $G$-representations, with $W$ irreducible. Prove that the following map $V\to V$ is a $G$-equivariant projection onto the $W$-isotypic component of $V$:
\[
\pi_W := \frac{\dim W}{|G|}\sum_{g\in G}\ba{\chi_W(g)}\cdot g.
\]
(\emph{In the case where $W$ is the trivial representation, this is Proposition 9}.)

\item[2.] Prove the last statement in Corollary 16. That is, given that the irreducible characters form an orthonormal basis of $\C_{cl}(G)$ with respect to the Hermitian inner product $(\cdot,\cdot)$, show that
\[
\sum_{\chi: \text{irred}} \ba{\chi(g)}\chi(h) = \begin{cases} 
|C(g)|, & \text{if } g\sim h,\\
0, & \text{otherwise},
\end{cases}
\]
where $C(g)$ is the centralizer of $g$ in $G$.

(\emph{Hint: use properties of unitary matrices}.)



\item[3.] Let $G$ and $H$ be finite groups. If $V$ is an irreducible representation of $G$ and $W$ is an irreducible representation of $H$, show that $V\otimes W$ is an irreducible representation of $G\times H$ (i.e. define an action of $G\times H$ on $V\otimes W$ and show that the resulting representation is irreducible). (This is called the \emph{external tensor product}; don't confuse it with the \emph{internal tensor product} we've already looked it.) Show that every irreducible representation of $G\times H$ is of this form. 

(\emph{Hint: use character theory}.)


\item[4.] Prove \emph{directly} that the induced representation given in Definition 17b is unique up to isomorphism. That is, fix two sets of coset representatives $\sigma_1,\ldots,\sigma_k$ and $\sigma_1',\ldots,\sigma_k'$ for $G/H$ with $\sigma_i^{-1}\sigma_i'\in H$. Let
\[
V = \bigoplus_{i=1}^k W_i, \qquad W_i = \{w_i|w\in W\},\qquad \text{with action } g\cdot w_i := (h\cdot w)_j, \text{ where } g\sigma_i = \sigma_j h,
\]
\[
V' = \bigoplus_{i=1}^k W_i, \qquad W_i' = \{w_i'|w\in W\},\qquad \text{with action } g\cdot w_i' := (h'\cdot w)_j, \text{ where } g\sigma_i' = \sigma_j' h'.
\]

Give an explicit $G$-isomorphism $V\to V'$.


\item[5.] There is another definition of induced representation that is more natural in some settings. Fix an $H$-representation $(\rho,W)$. Let $\Ind_H^G\rho$ refer to the definition from class of the induced representation, corresponding to a fixed set $\sigma_1,\ldots,\sigma_k$ of (left) coset representatives for $G/H$. Let
\[
\ind_H^G \rho = \left\{ f:G\to W | f(hg) = \rho(h) f(g) \text{ for all } h\in H, g\in G\right\},
\]
with $G$-action given by
\[
g\cdot f(g') := f(g'g).
\]

\begin{enumerate}
    \item Prove that $\ind_H^G\rho$ is a $G$-representation. Specifically, prove that $g\cdot f(g') := f(g'g)$ is a $G$-action on the space of functions $G\to W$, and that if $f\in\ind_H^G\rho$, then so is $g\cdot f$.
    \item Prove that $\Ind_H^G\rho\cong \ind_H^G\rho$ by showing that
    \[
    w_i\mapsto f_{w,i}, \qquad \text{where} \qquad \underset{(h\in H)}{f_{w,i}(h\sigma_j^{-1})} = \begin{cases} h\cdot w, & \text{if } i=j, \\ 0, & \text{otherwise}.\end{cases}
    \]
    is a $G$-isomorphism.
\end{enumerate}

\item[6.] A \emph{virtual representation} over $\C$ of a finite group $G$ is a formal integer linear combination of irreducible $G$-representations (if all coefficients are nonnegative, this is an honest $G$-representation). Likewise, a \emph{virtual character} is an element of $\C_{\t{cl}}(G)$ with integer coefficients in the basis of irreducible characters.

\begin{enumerate}
    \item If $\chi$ is a virtual character and $(\chi,\chi)=1$, prove that (exactly) one of $\pm\chi$ is an irreducible character.
    \item In the language of Lecture 9, fix a one-dimensional representation $\nu:L^\times\to\C$, let $\alpha = \nu|_{K^\times}$, and let $\rho_\nu$ be the virtual representation
    \[
    \rho_\nu := 
    \widetilde{\rho}_1\otimes\rho_{\alpha_1} -  \rho_{\alpha_1} - \Ind_{L^\times}^G \nu.
    \]
    Compute the virtual character of $\rho_\nu$, and prove that it is irreducible.
\end{enumerate}

\item[7.] Again use the setting of Lecture 9, and recall the Weil group $W := L^\times \rtimes \langle\sigma\rangle$, where $\sigma^2=1$ and $\sigma\ell = \ba{\ell}\sigma$ ($\sigma$ can be thought of as the nontrivial automorphism of $L$ fixing $K$).

Let $\tau$ be a two-dimensional representation of $W$, and write $\tau|_{L^\times} = \nu_1\oplus\nu_2$.
\begin{enumerate}
    \item If $\tau$ is irreducible, show that $\ba{\nu_1} = \nu_2$, and both $\nu_1$ and $\nu_2$ do not factor through the norm map $N_{L/K}$.
    \item If $\tau$ is reducible, show that $\nu_1$ and $\nu_2$ factor through $N_{L/K}$.
\end{enumerate}

\end{enumerate}

\end{document}
